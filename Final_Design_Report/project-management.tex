\section{Project Management}

\subsection{Team Roles}
%{\Large \textbf{Team Roles}}

\addcontentsline{toc}{section}{Team Roles}
\section*{Team Roles}

% \begin{minipage}[t]{0.4\textwidth}
%     \begin{itemize}
%     \item Column 2 content 1
%     \item Column 2 content 2
%     \end{itemize}
%   \end{minipage}

\begin{center}
\resizebox{\textwidth}{!}{
\begin{tabular}{ |c|c|c|c| } 
    \hline
    Name & Specialty & Responsibilities & PDR Contributions \\
    \hline
    Jackson Artis & CAD Work & cell3 & cell4\\
    \hline
    Morgan Baker & Machining & cell3 & cell4\\
    \hline
    Sam Dale & Coding & cell3 & cell4\\
    \hline
    Alexandra Koskosidis & Team Leader, Analysis & cell3 & cell4\\
    \hline
    Alex Rogers & CNC & cell3 & cell4\\
    \hline
    Evan Quinn & Electrical Engineering & cell3 & cell4\\
    \hline
    
\end{tabular}}
\end{center}

\vfill % Fill the rest of the page with whitespace
\newpage

\subsection{Personnel}
Alexandra Koskosidis: Team Leader, Coding\\

    It was the primary responsibility of the Team Leader to direct the individual members of the group and the direction of the project as a whole. The Team Leader worked with the members of her group to ensure all necessary tasks were undertaken and completed by delegating sub-projects and setting completion date objectives. She scheduled team meetings, posed concerns to the group for discussion, and updated all members on progress. Additionally, the role of Coding Specialist was fulfilled by writing, debugging, and managing the completion of code for all autonomous parts of the SaRR. \\
    
Alex Rogers: CNC, Hardware Systems\\

    The CNC Specialist was mainly responsible for CNCing all CAD parts by creating mill volumes and designing tool paths. The CNC Specialist worked closely with Al to check pieces and reserve space in the CNC. Thus, he was also very involved in the main CAD model, adjusting dimensions of parts and making final assemblies to accommodate design changes. The role of Hardware Systems Specialist consisted of organizing electronics, mounting sensors, debugging potential motor and sensor issues and ensuring proper construction of the SaRR. This role worked closely with the Coding Specialist to assist with electronics and sensor readings. \\
    
Evan Quinn: Manufacturing\\

    The Manufacturing Specialist was tasked with manually machining parts such as the axles, couplings, gearboxes, and gears, and working in tandem with the Hardware Systems Specialist to assemble the robot and connect motors and other electrical components. He worked closely with the Coding Specialist to ensure the functionality of the proximity and light sensors.\\
    
Jackson Artis: CAD Modeling\\
    
    The CAD Specialist oversaw the CREO modeling process, in which design ideas were transformed into CAD models to allow for clear dimensioning and subsequently accurate manufacturing of parts. The CAD specialist was responsible for making sure the CREO model was in a state that was easily and readily transferable to the manufacturing process. What's more, he was responsible for aiding in and overseeing all CREO simulations and analyses including various buckling analyses and center of gravity calculations. \\ 

Morgan Baker: Analysis, Manufacturing Assistant\\

    The Analyst was responsible for completing free-body diagrams, calculations regarding torque, center of gravity, static stability margins, and stress on crucial components to ensure that the team avoided major mechanical failures. The Manufacturing Assistant was responsible for assisting with the machining of and overseeing of all machined parts. The Manufacturing Assistant worked closely with the Manufacturing and CNC Specialists to ensure that the best possible methods were used to create parts.\\
    
Sam Dale: Design, Electronics\\

    The Design Specialist focused on the development and implementation of mechanical design and functionality, primarily during the initial design phase of the project. While design decisions were ultimately made in full-team brainstorming sessions, the Design Specialist spearheaded these initiatives and oversaw the creation of the initial CAD models. The Electronics Specialist role included planning out electronic placement, soldering and wiring all the components, ensuring that they were connected properly, and debugging electronics issues as they arose.
    

\subsection{Schedule and Tasks}

The schedule for the whole project is outlined below, with approximate hours spent on each task and commentary about success provided.

\begin{itemize} 
\item Course Milestone: Demonstrate Open-Loop Drive-train (24:35)
\begin{itemize}
\item CNC motor holders, cut axles, assemble belt tighteners
\item Wired breadboard, soldered wires
\item Programmed controller, connected to breadboard
\item The initial wiring was not done by the team, but rather by members of a lab group, and was not considered up to the SaRRchaeologists' standards. Most of the wires were soldered again, and connections were remade.
\item Due: Week of Feb. 25th
\item Completed: March 1, 2019--On time
\end{itemize}
\end{itemize}
\begin{itemize}
\item Course Milestone: Closed-Loop Navigation to Light Source (52:05) 
\begin{itemize}
\item Attached light and proximity sensors 
\item Connected sensors to outputs, read sensors 
\item Wrote code to read sensor outputs respond appropriately
\item Due: Week of Mar. 4th
\item Completed: March 7th, 2019--On time
\end{itemize}
\end{itemize}
\begin{itemize}
\item Team Milestone: Finalize Design, Parts Order
\begin{itemize}
\item Individual brainstorming, idea proposals and deliberation
\item CAD model and cardboard prototyping
\item Bill of Materials
\item Initial parts order
\item Motor torque and gearing calculations
\item Due: March 11
\item Completed: March 24th, 2019--Later than desired (partly due to Spring Break)
\end{itemize}
\end{itemize}
\begin{itemize}
\item Team Milestone: Begin Assembly (84:30)
\begin{itemize}
\item Manufactured gearbox components 
\item Calculations for center of gravity, lifting arm torque, energy use, etc.
\item Due: Week of March 24th
\item Completed: On time. Sufficient work was completed to declare this task fulfilled
\end{itemize}
\end{itemize}
\begin{itemize}
\begin{itemize}
\item Course Milestone: PDR (50:40)
\begin{itemize}
\item Due: Apr. 4
\item Completed: Apr. 4, 2019--On time
\end{itemize}
\end{itemize}
\item Course Milestone: Speed Trial (228:00)
\begin{itemize}
\item Finished chassis assembly
\item Assembled gearbox, mounted motors, connected wheels 
\item Attached and secured hardware
\item Due: Apr. 5
\item Completed: Apr. 8, 2019--One day late
\item The Speed Trial milestone was met the Monday after the due date. The deadline was missed due to the shearing of the plastic driving gears several hours before the trial.
\end{itemize} 
\end{itemize}
\begin{itemize}
\item Course Milestone: Open-Loop Wall Traversal (108:30)
\begin{itemize}
\item Manufactured lifting arms 
\item Mounted lifting motor and driving sprockets
\item Programmed controller
\item Due: Apr. 12
\item Completed: Apr. 12, 2019--On time
\end{itemize}
\end{itemize}
\begin{itemize}
\item Course Milestone: Open-Loop Object Retrieval/Placement (70:30)
\begin{itemize}
\item Machined medkit arm
\item Mounted and attached medkit motor 
\item Due: Apr. 19
\item Completed: Apr. 19, 2019--On time
\end{itemize}
\end{itemize}
\begin{itemize}
\item Course Milestone: Closed-loop Object Placement (92:00)
\begin{itemize}
\item Mounted light sensors from sample robot
\item Modified sample robot code to work with DynaSaRR
\item Began programming for autonomous chute navigation
\item Due: Apr. 26
\item Completed: Apr. 26, 2019--On time
\end{itemize}
\end{itemize}
\begin{itemize}
\item Team Milestone: Closed-loop Chute Navigation (76:30) 
\begin{itemize}
\item Mounted side proximity sensors
\item Completed code for closed-loop chute navigation
\item Some issues were encountered due to unpredictable performance as battery charge decreased and right rear wheel motor unreliability
\item Due: May 3
\item Completed: May 3, 2019--On time
\end{itemize}
\end{itemize}
\begin{itemize}
\item Team Milestone: Autonomous Wall Traversal, Successful Course Completion, Repair Mechanical Issues (146:34)
\begin{itemize}
\item Wrote and debugged autonomous wall traversal code
\item Joined autonomous portions of code
\item Repaired mechanical issues with permanent solutions to ensure proper functionality going into competition
\item Due: May 10
\item Completed: N/A
\item The SaRRchaeologists were unable to produce reliable code for the autonomous portion of the course. While each individual component worked individually (and the chute and light/placement codes were able to work in tandem), the unreliability of robot performance during testing (due to dropping charge and mechanical problems) prevented the completion of robust code.
\end{itemize}
\end{itemize}
\begin{itemize}
\item Course Milestone: Demo Day, May 14th
\begin{itemize}
\item Demonstrated search and rescue capabilities of DynaSaRR
\item Unfortunately, the results of this demonstration proved to be disappointing. Successes, failures, and potential explanations are discussed in section 7.
\end{itemize}
\end{itemize}
\item Course Milestone: FDR
\item Due: May 14
\item Completed: May 14th, 2019--On time

\subsection{Key Tasks with Leaders}

\begin{table}[htv]
\resizebox{\textwidth}{!}{%
\begin{tabular}{lll}
\textbf{Key Tasks}                          & \textbf{Task Leader} & \textbf{Completed} \\
Autonomous Light Navigation--Sample Robot   & Sam Dale             & 3/7/19             \\
Finalize Preliminary Design                 & Sam Dale             & 3/29/19            \\
Assembly of SaRR                            & Evan Quinn and Alex Rogers          & 4/4/19             \\
PDR                                         & Jackson Artis        & 4/5/19             \\
Speed Trial                                 & Evan Quinn & 4/8/19             \\
Open-Loop Wall Traversal                    & Alex Rogers          & 4/12/19            \\
Open-Loop Medkit Retrieval and Placement   & Morgan Baker          & 4/19/19            \\
Closed-Loop Light Tracking and Medkit Placement & Sam Dale             & 4/26/19            \\
Closed-Loop Chute Navigation                & Alexandra Koskosidis & 4/30/19            \\
Closed-Loop Continuity                      & Alexandra Koskosidis & 5/10/19            \\
Final Presentation                          & Morgan Baker         & 5/13/19            \\
FDR                                         & Jackson Artis        & 5/14/19           
\end{tabular}%
}
\end{table}



\subsection{Management Approach}

The team recognized that the most important part of project management is constant, effective communication. Before any design or manufacturing began, the group set up a chat on Facebook Messenger, which allowed all members to collaborate at the same time from their phone or laptop, and actively identify who has seen the messages and who has not (unlike text message). Furthermore, a WhenIsGood form was compiled to identify the time slots during which each team member would be available. This form made it easy for the Team Leader to organize team meetings and assign work to people by the hour. Team meetings were scheduled during the l period every Monday, which was considered the Team Lab Time. Any other meetings were scheduled as needed. Team members recorded their work hours in the Time Sheet after any time spent working on the project. This log included a description of the task worked on, which held team members accountable for their work and allowed for an ongoing tally of the hours spent on each task and on the project as a whole.

Initially, tasks were managed rather informally, with team members sending the times they planned to work that week into the group chat and the Team Leader assigning them the appropriate tasks based on that scheduling. Unfortunately, this strategy proved inefficient, as high-priority tasks assigned in a certain order would end up being completed later than expected due to unforeseen changes in team members' availability, and therefore delayed subsequent tasks.

Following the SaRRchaeologists' failure to meet the speed trial milestone on time, a long team meeting was held in which each team member was encouraged to speak freely and bluntly (though constructively) about challenges faced and mistakes made by each person, as well as suggestions for improvement, which included comments on work assignment strategies, disorganization and quality of work. This conversation was helpful for discussing major themes that Frank Ryle presented in lecture such as conflict approaches, risk analyses, and viewing group members as stakeholders that were near as important as other stakeholders like Professor Nosenchuck and Aaron. As a result of this discussion, the management strategy changed completely for the rest of the semester. This meeting saw the creation of an organized, color-coded To Do List, in which tasks could be entered and deadlines assigned. The list had two main features that drastically improved the efficiency of the group's organization: first, tasks could be assigned to individual team members, and comments could be recorded next to the task to make the information easily accessible. Second, the list could was sorted by priority, and the dependency of certain tasks on those before them was made clear. This proved to be much more successful, as it placed the onus on individual team members to see their upcoming assignments and plan their work hours accordingly. Additionally, it allowed for anyone finishing their assignment early to immediately begin work on the next task, rather than having to discuss in the group what was next to be done.


A major issue anticipated by the SaRRchaeologists was manufacturing and coding errors made as a result of fatigue and carelessness after many hours of work. As a result, team members attempted to work in pairs or groups of three whenever possible so that a partner could provide a "sanity check" and catch any overlooked problems in the manufacturing of a part or the writing of code. On the other hand, with the exception of design discussions, the SaRRchaeologists tried to avoid having more than two or three members working on the same task at once, as this frequently resulted in one or more members watching the others complete the task and making inefficient use of their limited time.

Any personal issues or friction that arose were resolved with promptly--team members were open with each other and communicated in a direct, constructive way, pointing out any practices that were not ideal and suggesting improvements and ways to remedy mistakes, while also acknowledging strengths and successes. Smaller design problems were generally resolved on the spot as they arose by whichever team members were working on that task, but larger changes were always brought to the whole group for discussion, and not implemented until all 6 group members were fully on board. If someone disagreed with a design decision or preferred another method, discussion continued until they felt comfortable with the solution and were ready to help implement it. Overall, the group worked cohesively, with no major issues and successful communication.

\clearpage

\subsection{Total Time}
% \begin{center}
% \begin{tabular}[ |c|c|c| ]
%     \csvautotabular{./resources/total-time_v1.csv}
%     \caption{Total time spent working on the project for each team member.}
% \end{tabular}
% \end{center}

\begin{table}[htv]
\centering
\begin{tabular}{l|l|l|}
\hline
\multicolumn{1}{|l|}{\textbf{Team Member}} & \textbf{Hours Worked} & \textbf{Percent of Total} \\ \hline
\multicolumn{1}{|l|}{Alexandra Koskosidis} & 185:50 & 19.67\% \\ \hline
\multicolumn{1}{|l|}{Alex Rogers} & 142:20 & 10.07\% \\ \hline
\multicolumn{1}{|l|}{Evan Quinn} & 95:30 & 10.11\% \\ \hline
\multicolumn{1}{|l|}{Jackson Artis} & 139:25 & 14.76\% \\ \hline
\multicolumn{1}{|l|}{Morgan Baker} & 140:00 & 14.82\% \\ \hline
\multicolumn{1}{|l|}{Sam Dale} & 241:30 & 25.57\% \\ \hline
 & 944:35 & 100\% \\ \cline{2-3} 
\end{tabular}
\caption{Total time spent working on the project for each team member.}
\end{table}


\clearpage
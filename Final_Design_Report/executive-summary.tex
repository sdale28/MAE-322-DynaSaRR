\section{Executive Summary}
In light of the September 11th Attacks, this course sought to give students a chance to apply their theoretical knowledge to solve a real-world problem by designing and prototyping a robot that would be able to navigate a disaster zone in order to provide aid to survivors. Cursory analysis of some of America's most vital rescue missions reveals that planning in advance of the situation is indisputably necessary. As such, the SaRRchaeologists set out to create a robot that would demonstrate the group's ability to combine careful design and problem-solving with innovative thinking.

The SaRRchaeologists built a Search and Rescue Robot (SaRR) with the objective of navigating an obstacle course with challenges including: manually navigating to and picking up a 4" x 4" x 4" medical kit (which will be referred to as "the medkit" for the remainder of this report) weighing 3 pounds, autonomously traversing a 12" high wooden wall, autonomously navigating a 3' chute with sharp bends, and finally, autonomously tracking a light source to place the medkit in a 4" deep basket, mounted on top of a 4" block of wood containing the light source. 

DynaSaRR's design involved careful prior assessment of the situation, as well as an innovative approach to the problem. The design uses a novel mechanism to lift it over the wall: 13" arms (the "lifting arms") which rotate to raise the front wheels, while the 6" rear wheels propel DynaSaRR forward onto each of the two steps. The arms then rotate around once again and hook onto the back of the wall, pulling the robot over with the help of the rear-wheel drive. As DynaSaRR dismounts from the wall, the combination of the forward-angled front wheels, shock-absorbing springs, and force-distributing frame all work in unison to lessen the effect of the sudden impact of hitting the ground. These components were designed in order to ensure the robustness of the robot in the face of various falls, collisions, and other unforeseen damage. Additionally, DynaSaRR's chassis sits six inches above the ground in order to avoid debris and to minimize the required length of the lifting arms. 

DynaSaRR represents the results of careful planning and attention to detail. In anticipation of inevitable design changes and modifications over the course of the building and testing process, DynaSaRR was designed to be easily modifiable. The frame was made of 80/20 extrusion to allow for easy mounting and adjustment of parts, and the drive train and controlling electronics were nested between two layers of acrylic so that subsequent work would cause little to no disruption of the drive train and main operations hub. 


\newpage
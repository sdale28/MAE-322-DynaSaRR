\section{Further Work and Conclusions}

DynaSaRR's base was designed so that it could be adapted to any scenario or obstacle. This particular iteration of DynaSaRR was designed so as to best handle the specific objectives for this course: retrieving a medkit, traversing a wall, navigating a chute, and depositing  the medkit. The SaRRchaeolgists carefully analyzed the objectives in order to create the pieces necessary for DynaSaRR's success. What's more, the SaRRchaeologists sought to create a robot that would be robust enough so as to assuage any fear of breakage either prior to or during the navigation of the course. 

The SaRRchaeologists were, undoubtedly, successful in their attempt to create a sturdy and robust robot. By looking at the high stress subsystems, this robot would buckle only if it were to encounter an addition 150 lbf \ref{fig:buck2}. What's more, the initial drop test before the final testing proved DynaSaRR could handle the shock of a drop. Finally, throughout the building process, DynaSaRR experienced many falls, from many angles, and still was able to withstand the hits with little to no damage. 

The SaRRchaeologists were, however, unsuccessful in their complete navigation of the course. It is important to note, however, that DynaSaRR was more than capable of navigating the course while being driven in open-loop mode. Time and time again, the SaRRchaeologists were able to succeed at each and every objective individually. In this respect, it can be said that the SaRRchaeologists were successful in their hardware manufacturing and their overall design.

Ultimately, the closed loop caveat of this task proved to be the downfall of DynaSaRR and the SaRRchaeologists. In spite of being able to successfully traverse the wall and deposit the medkit countless times using open loop, the code used for the closed loop driving proved insufficient. The lack of robust code could be attributed to many things, but the most salient reason was most likely the SaRRchaeologists' simple inability to spend enough time working on the code in the competition conditions. There were many mechanical issues that consumed enough time that the SaRRchaeologists had to continually push back their testing start times. This shortened the amount of time available to troubleshoot and improve the closed-loop code.

Given more time, the SaRRchaeologists would first attempt to find a means to power DynaSaRR such that it could operate at full power for longer. The diminished battery power caused the SaRRchaeologists to have to cease tests sooner and with greater frequency then desired. Subsequently, the group would spend a significant amount of time troubleshooting the code, but the varying performance depending on battery charge ultimately proved to be insurmountable for the SaRRchaeologists within the given time constraints. 



